\documentclass[a4paper, 12pt]{article}

\usepackage[utf8]{inputenc}
\usepackage[T1]{fontenc}
\usepackage{lmodern}
\usepackage[english]{babel}
\usepackage{amsmath}
\usepackage{amsfonts}
\usepackage{amssymb}
\usepackage{graphicx}
\usepackage{hyperref}

\title{Report: Reliable, Safe, and Secure Software}
\author{Andrea Ciccarello}
\date{\today}

\begin{document}

\maketitle

\begin{abstract}
This report describe the work A. Ciccarello (AC) has done in the Isolette project.
\end{abstract}

\tableofcontents

\section{Isolette Control System Requirements}

I worked with the team in order to write every requirements related to the CLT software unit in the project,
due to the nature of the CRL requirements we had to collaborate with every team in order to have a clear
separation of each module in the project.
Each requirement has been written in strict docs format and has been reviewed by the team.

The requirements are divided in High Level Requirements (HLR) and Low Level Requirements (LLR),
each LLR must have one or more HLR parent (Traceability) and each HLR must have one or more LLR child.
CLT has been build as a state machine, main states are: CTL\_INIT, CTL\_INPUT, CTL\_OPERATION, CTL\_ERROR.

In each state the CTL has handle multiple events and check in order to operate in a safe and secure way.

We have defined 19 Low Level Requirements and 4 main High Level Requirements for the CLT unit

\begin{itemize}
    \item CTL shall control temperature
    \item CTL shall assess system health
    \item CTL shall provide notifications and alarms
    \item CTL shall solicit and respond to user interaction
\end{itemize}

CTL have to interact with every unit of the project, handle correctly self test of the system, report errors to the UI unit.

\section{Assumptions}

I helped writing about Assumptions related to nurses. 
This is the current list available in the project paper (Isolette\_nurse\_assumptions.csv):

\begin{itemize}
    \item Nurses are highly specialized professionals, trained to provide expert care to Infants in intensive care.
    \item Nurses can be trained to use technical devices, including those that are controlled by computers with ordinary user interface elements.
    \item Nurses are not visually or hearing impaired and can interact with ordinary user interface elements such as buttons and keyboards.
    \item Nurses can move Infants as needed.
\end{itemize}

For each nurse assumptions I alswo wrote about the reasons why we have to consider them relevant for the project.

\section{Program Documentation and Doxygen}

I worked on the Doxygen and program documentation, ensuring the correctness of the current README instructions for the generation of the documentation.
I studied how to write Doxygen documentation and provided assistance in using special commands properly. 
I collaborated with the CMake team to enable the use of make for generating documentation for Test and System source code; previously, this was only possible with a script. 
I contributed to the current Doxygen guidelines, making multiple commits to ensure the project is up to date with the latest standards.
I deleted useless command like @breaf and @par that were misused in the project.
I worked with the Traceability man to ensure a common guidelines for the documentation of the project.

\section{Hardware Abstraction Layer: Isolette and Room Air}

I reimplemented the Hardware Abstraction Layer source code for Isolette Air and Room Air. I addressed issues identified in the mutation analysis and collaborated with the Heater team to establish a common method for handling temperature and heat exchange between heating surfaces and the air.
Isolette and Room air Hardware Abstraction Layer now have 100\% mull Mutation score and MCDC coverage.

\subsection{Isolette Air HAL}

Isolette Air HAS has to interact with the Room air and indirectly with the heater surface temperature in order to ensure
that the heat can be injected in the Isolette and the temperature can be controlled in a safe way.

Fourier law of heat conduction allows to compute the heat transfer rate through the isolette material.

\begin{equation}
q = k A \frac{T_2 - T_1}{L}
\end{equation}

where:
\begin{itemize}
    \item q is the heat transfer rate in watts (W),
    \item k is the thermal conductivity of the material (\( W/mK \)),
    \item A is the cross-sectional area of the material (\( m^2 \)),
    \item T1 and T2 are the temperatures at the ends of the material (K),
    \item L is the thickness of the material (m).
\end{itemize}

\subsection{Room Air HAL}

The implementation of the Room Air is simpler because there is not function that dynamically update it like the Isolette Air.
The Room Air is always at the same temperature because we have assumed that there is an external system that has this task.

Both HALs have a saturation mechanism, it ensures that the temperature is always in a safe range.

\section{Comparison with FAA: Requirements and presentation}

I worked with the FAA analysis team to compare the Isolette project with the FAA project report developed by the US government. This analysis was presented to the entire team.

My responsibilities included studying:

\begin{itemize}
    \item The project description of "External Entities" and their specifications:
    \begin{itemize}
        \item Temperature Sensor
        \item Heat Source
        \item Operator Interface
    \end{itemize}
    \item Safety requirements, such as prolonged exposure of infants to unsafe heat or cold, classified as a catastrophic event:
    \begin{itemize}
        \item Thermostat System Function
        \item Regulate Temperature Function
    \end{itemize}
\end{itemize}

I also had to write a report about requirements of the FAA project inside a chapter of the paper.
List of every requirements of the FAA Isolette project can be found in /Isolette/FAA/Specification/Isolette.sdoc

\section{MISRA C:2012 Compliance}

We started to work on the MISRA C:2012 compliance of the project with the study of the MISRA C:2012 guidelines.
Every MISRA C guideline is classified as either being a “rule” or a “directive”.
A directive is a guideline for which it is not possible to provide the full description necessary to perform a check for compliance.
A rule is a guideline for which it should be possible to check that source code complies without needing any other information.

Guidelines can be classified as:
\begin{itemize}
    \item Mandatory: A guideline that must be followed.
    \item Required: A guideline that should be followed.
    \item Advisory: A guideline that is only recommended (can't just ignore them).
\end{itemize}

For each rule we have a brief description, a category, the version which the rule applies to, the scope of the analysis (like Single Translation unit), 
an Amplification (not mandatory), an Rationale where is explained why the rule is important, a list of examples (not mandatory).
For this project we use Eclair \cite{Eclair} as a tool static analysis tool to check the MISRA C:2012 compliance of the project.

We provided a list of Project deviations from MISRA-C and C++ guidelines, it is available in the dedicated paper chapter.
These deviation are implemented via a change in current Eclair configuration thanks to Bugseng team.

\section{Logging}

I studied the current logging system, handled by the C++ Boost library, available in the project. 
I wrote a tutorial on this topic, included in an appendix of the project documentation. Additionally, I provided a brief description of the current logging guidelines and assisted in applying them to the project.
I improved the current logging initialization system in the main.cc file with a better handling of the log file and when we have to set the 
severity level of the log system.

\subsection{Logging guideline}

I have ensure a simple logging guideline for the project:

\begin{itemize}
    \item Use the LOG macro to log messages.
    \item Start each log message with a capital letter.
    \item End each log message with a period.
    \item Use the severity levels properly and don't abuse the info level.
\end{itemize}

Syntax of the LOG message is checked by a python script that I run locally.

\subsection{Logging abuses}
In the project there was an abuse of the info logging level, this is not a good practice because when debugging in case of an anwanted execution path we have to search for the error in a lot of info logs.
The project has to prioritize the use of trace and debug logging error with a zero suppression policy.

For example in the Isolette Air HAL every tick the current temperature is logged with the trace level and only every 50 ticks the info level is used.

\begin{thebibliography}{9}
    \bibitem{latexcompanion} 
    Leslie Lamport.
    \textit{LaTeX: A Document Preparation System}.
    Addison Wesley, 2nd edition, 1994.

    \bibitem{Eclair}
    Bugseng Website
    https://www.bugseng.com/eclair/eclairit

    \bibitem{Doxygen}
    Doxygen Manual
    https://www.doxygen.nl/manual/index.html

\end{thebibliography}

\end{document}
